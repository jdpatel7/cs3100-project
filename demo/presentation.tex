\documentclass[12pt]{article}

\usepackage{enumitem, amsmath, amssymb, parskip}

\setcounter{secnumdepth}{0} % sections are level 1
\reversemarginpar

\begin{document}
{
	\parindent=0pt
	\bfseries 
	Computer Science 3100 \hfill Winter Semester 2019 \\
	Dr.~Rod~Byrne \hfill J.~House, G.~Mensah, J.~Patel\\
}

\vskip 1pc

\begin{center}
	\LARGE\bfseries Presentation Outline
\end{center}

\section{What is the {\sffamily React} Library?}\marginpar{Gillian}
\begin{enumerate}[label={(\arabic*)}]
	\item {\bfseries HTML: What HTML?}

		{\sffamily React} projects serve one HTML file, \verb|/index.html|. [Show this file.] This file includes one (rather large) JavaScript file created by WebPack which encompasses all the modules and libraries used in the app. 

	\item {\bfseries JSX: Adding logic to markup.}
		React's ``JavaScript'' is more than vanilla JavaScript. Importing {\ttfamily React} from the {\ttfamily react} library allows us to write markup HTML elements (and custom derivatives) in JavaScript --- no we don't mean huge strings containing the markup. [Show \verb|frontent/src/components/PostCreatePage.jsx|.] These files are commonly saved with a \verb|.jsx| extension.

		These can be objects, extending {\ttfamily Component} from the {\ttfamily react} library --- in which case they can contain {\em state} --- or they can be functions, which do not have state. These {\em components}, as they are called, allow for modularity.

	\item {\bfseries {\sffamily React} Router: Route Awareness on the Frontend.}

		{\sffamily React} is a frontend JavaScript library that creates a Single Page Application --- the first page load is the only one where the browser receives HTML and the page ``flashes''; subsequent requests fetch at most a JSON object. To allow a logical flow, and to make pages bookmark-able, SPAs need to manipulate the URI in the address bar of the browser.

		We use a library called {\ttfamily react-router-dom} to do this. It pushes ``pages'' into the history stack as the user navigates the website, and includes JavaScript to get the browser URI and render a particular DOM layout sort-of dynamically based on the route. [Show \verb|frontend/src/App.jsx|.]
\end{enumerate}

\section{Build Features}\marginpar{Jacob}
\begin{enumerate}[label={(\arabic*)}]
	\item {\bfseries {\sffamily Node} Inception: The Project within a Project.}
		
		[Begin with view of project structure.] We have two {\sffamily Node} ``projects'': the frontend and backend. All of our {\sffamily React} codebase is in the frontend. [Show {\ttfamily react} in \verb|frontend/package.json|.] All of our {\sffamily Express}, {\sffamily SQLite}, etc. code is in the backend. [Show {\ttfamily express} in \verb|package.json|.] 

		How do we combine these? We run \verb|make build| which installs all required packages in the frontend and backend, and then runs WebPack to export a production-ready {\em static} site (HTML, JS, images, fonts, etc.) to \verb|frontend/build/|. [While that is compiling, open \verb|app.js|.]
		Our {\sffamily Express} app, based in \verb|app.js|, begins with routes we know exist: \verb|/robots.txt| for site crawlers, \verb|/api/*| for all of our asyncronous frontend-backend JSON communication (as well as image serving!), as well as anything in the \verb|frontend/build/| folder like static JS, font, and image assets.

		But the {\sffamily React} router includes routes such as \verb|/login| which the {\sffamily Express} router doesn't know about! We need hits to the web server for such routes to always serve \verb|frontend/build/index.html| so we use one final {\sffamily Express} wildcard route for this.


		\fbox{\bfseries START THE APP}

	\item {\bfseries {\sffamily Helmet}: Express App Hardening}

		[Ensure the {\sffamily Express} default app is running on port 8000.] If we run \verb!curl -i localhost:8000 | less!, we see the headers that {\sffamily Express} sends to clients. Running \verb!curl -i localhost:3100 | less! shows {\em our} instance of {\sffamily Express}' headers.

		Security through obscurity is no security at all, but advertising your attack surface unnecessarily isn't advisable either.

		\fbox{\bfseries OPEN APP IN BROWSER AND INTRODUCE IT}

	\item {\bfseries Web Sockets: Pulling it all Together}

		As we can see, our test user has a few unread notifications. As stated in class, updating such a list of real-time events could mean poling a server. This is neither elegant nor scalable. So we used Web Sockets. 

		One thing we did differently is this\ldots\ [Open \verb|frontend/src/App.jsx|.] On the browser {\em close} event, we can tell that our socket has disconnected for some reason --- perhaps network access was lost. When this occurrs, we open a {\em new} connection to the server, without the user needing to refresh the page and without any major user experience disruption.
\end{enumerate}


\section{Demo}\marginpar{Jay}
	\begin{enumerate}[label={(\arabic*)}]
		\item {\bfseries Gatekeeper: Securing {\sffamily NumHub}}

			Due to time constraints, we won't show registration, just login with pre-made demo users.

			\fbox{\bfseries OPEN SECOND WINDOW OF THE APP}

			If a user forgets their username [Demonstrate this.], we show them that they made a mistake. If they forget their password, we allow an option to reset it by answering a pre-defined security question [Demonstrate this but do not proceed; go back to password form.]. 

			If they were to attempt to sign in 10 times (our test user already tried 8 times), a user locks their account. From this point, only an administrator can reset their logon attempt count. [Go to the first window and demonstrate this, then log on in the second window with the now-reset account.]

		\item {\bfseries ``I have a question!'': Creating new Posts}

			We've split the set of all posts on our website intoa collection of disjoint subsets. In other words, we have separate ``sub-sites'' for primary school questions, elementary school, middle schol, high school, undergraduate, and graduate-level questions. 

			So, all the questions we see here are at the {\em Undergraduate} level. If we expnd one of these questions and open it, notice the styling: \LaTeX\ and Markdown has been used to typeset the content. 

			When you post a new question [Open the post page paste in a question from the question document.], you will need to tag your question as well. This makes the site easier to use and makes finding relevant questions faster.

		\item {\bfseries Catering to the Know-It-Alls: Answering Questions}

			Of course we also have a way to answer questions posted to the site. [Go back to the home page and choose the tag for {\em Trigonometry} and tick {\em Unanswered}. The sample question should match this search.] 

			If we know an answer to a question on the site, it's easy to post it. But what if the clarity of an answer (or question) is impeded without visual assets? If we'd like to include a graph, for example. 

			[Put in the image for the answer to the trig question.]

			We've developed an API that can upload, compress, store, and retrieve images for this purpose. If we inspect the post preview, we see that the image references the \verb|/api| route of our site. 

			Each image that's uploaded is compressed on the server, hashed, and then a URI is created in {\sffamily Express} which maps this hash to a bas-64-encoded image stored in the database. As shown in class, {\sffamily Express} can easily serve images from a filesystem; doing so from another source is no different.


	\end{enumerate}





\end{document}
